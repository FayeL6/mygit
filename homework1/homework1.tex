\documentclass{ctexart}

\usepackage{graphicx}
\usepackage{amsmath}

\title{作业一: 罗必达法则的叙述与证明}


\author{李方圆 \\ 统计学 3190104914}

\begin{document}

\maketitle


这是一个数学分析领域的问题。罗必达法则是在一定条件下对分子和分母分别求导再求其极限来求一定未知值的方法。
\section{问题描述}
形式如下:
\begin{equation}\label{1}
  \lim\limits_{x \to x_{0} } \frac{f(x)}{g(x)} = \lim\limits_{x \to x_{0}} \frac{f^{'}(x)}{g^{'}(x)}
  \tag{1}
\end{equation}
\\
通常有 $\frac{0}{0}$ 型,$\frac{\infty}{\infty}$ 型。
\\
对于其他不定式如 $0 \cdot \infty$, $1^{\infty}$, $0^{0}$ 等,
通常可以经过简单变换,化为$\frac{0}{0}$型或$\frac{\infty}{\infty}$型。
\\
罗必达法则的使用条件:
\begin{itemize}
    \item [1)] 
    极限满足 $ \frac{0}{0} $ 或 $ \frac{\infty}{\infty} $
    \item [2)]
    f(x),g(x)在 $ x_{0} $去心邻域内可导,且 $ g^{'}(x) \neq 0 $
    \item [3)]
    $ \lim\limits_{x \to x_{0}} \frac{f^{'}(x)}{g^{'}(x)} = a $, (a为有限实数或无穷大)
\end{itemize}
下面给出罗必达法则的证明。
\section{证明}

\subsection*{$\frac{0}{0}$型}
假设
\begin{equation}\label{2}
  \lim\limits_{x \to x_{0}} f(x) = 0, \quad  \lim\limits_{x \to x_{0}} g(x) = 0 \tag{2}
\end{equation}
由\ref{2}及罗必达法则使用条件,我们有
\begin{equation}\label{3}
  f(x_{0})=0, \quad g(x_{0})=0 \tag{3}
\end{equation}
根据柯西中值定理,以及\ref{2}和\ref{3},存在 $m \in (x_{0},x)  $,使得
\begin{equation}\label{4}
  \frac{f(x)}{g(x)} = \frac{f(x)-f(x_{0})}{g(x)-g(x_{0})} = \frac{f^{'}(m)}{g^{'}(m)} \tag{4}
\end{equation}
令 $ x \to x_{0} $,则 $ m \to x_{0} $,即有
\begin{equation}\label{5}
  \lim\limits_{x \to x_{0}} \frac{f(x)}{g(x)} = \lim\limits_{m \to x_{0}} \frac{f^{'}(m)}{g^{'}(m)} \tag{5}
\end{equation}

\subsection*{$\frac{\infty}{\infty}$型}
此时
\begin{equation}\label{6}
  \lim\limits_{x \to x_{0}} f(x) = \infty, \quad \lim\limits_{x \to x_{0}} g(x) = \infty \tag{6}
\end{equation}
可得
\begin{equation}\label{7}
  \frac{f(x)}{g(x)} = \frac{\frac{1}{f(x)}}{\frac{1}{g(x)}} \to t, \quad x \to x_{0}. \quad t \neq 0   \tag{7}
\end{equation}
于是
\begin{equation}\label{8}
  \lim\limits_{x \to x_{0}} \frac{f(x)}{g(x)} =
  \lim\limits_{x \to x_{0}} (\frac{\frac{1}{g^{2}(x)}}{\frac{1}{f^{2}(x)}})
  (\frac{\frac{1}{f(x)}}{\frac{1}{g(x)}}) =
  \frac{1}{t^2} \lim\limits_{x \to x_{0}} \frac{(\frac{1}{f(x)})^{'}}{(\frac{1}{g(x)})^{'}} =
  \frac{1}{t^2} \lim\limits_{x \to x_{0}} t^2 \frac{f^{'}(x)}{g^{'}(x)}
  \tag{8}
\end{equation}

\begin{equation}\label{9}
  \lim\limits_{x \to x_{0}} \frac{f(x)}{g(x)} = \lim\limits_{x \to x_{0}} \frac{f^{'}(x)}{g^{'}(x)} \tag{9}
\end{equation}

证毕。

\end{document}

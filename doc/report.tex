\documentclass{ctexart}

\usepackage{graphicx}
\usepackage{subfigure}
\usepackage{amsmath}
\usepackage{amsfonts}
\usepackage{algorithm}  
\usepackage{algpseudocode}  
\usepackage{amsmath}
\usepackage[backref]{hyperref}
\renewcommand{\algorithmicrequire}{\textbf{Input:}}
\renewcommand{\algorithmicensure}{\textbf{Output:}}
\renewcommand{\abstractname}{Abstract}
\renewcommand{\refname}{Reference}

\title{作业五:安装最新版本的gsl}

\author{李方圆 \\ 统计学  3190104914}

\begin{document}

\maketitle

\begin{abstract}


\section{功能概述}
使用 \verb|brent| 法求解方程的根。(\verb|brent|方法为结合了二分法、割线法、逆二次插值法的一种求根方法)\\

使用\verb|roots.c|,可以针对性求解一元二次方程的根。\\

本例中,则求解了如下方程的根:
\begin{equation}
  x^{2}-5=0
\end{equation}
 
\section{部分运行框架与部分参数介绍}
\subsection{运行框架}
使用 \verb|demo_fn.h| 与 \verb|demo_fn.c| 自定义了一元二次函数的构造,并进行引用。\\

除此之外,根据需要引用了 \verb|gsl| 中的库。

\subsection{参数介绍}
根据 \verb|demo_fn| 的规定,使用三个参数 \verb|a,b,c| 产生需要的一元二次方程
\begin{equation}
  (ax+b)x+c
\end{equation}

规定了最大迭代次数 \verb|max_iter|。\\

\verb|T, s| 用来选择求根方法。\\

\verb|r_expected| 为根的精确值,用来计算误差。\\

\verb|x_lo, x_hi| 定义了定义域的上下界。

\section{运行原理}

使用 \verb|brent| 方法进行迭代求根,更新根、误差、根所在范围上下界,直到状态满足 \verb|GSL_SUCCELL| 中收敛的要求(若满足,输出Converged),或超过最大迭代次数,停止运行。





\end{document}
